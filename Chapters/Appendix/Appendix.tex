\begin{center}
\Large \textbf{Appendix: Formal Definitions and Proofs} \\[1ex]
\end{center} 

\noindent{\bf (Inheritance Network)} An \normalfont{inheritance network} $\mathcal{I}$ is a triple $\langle\mathcal{B},\sqsubseteq,\#\rangle$ where:
\begin{itemize}
\item $\mathcal{B}$ is a finite set of \normalfont{basic elements}
\item $\sqsubseteq\subseteq\mathcal{B}\times\mathcal{B}$ is the basic \emph{inheritance} relation
\item $\#\subseteq\mathcal{B}\times\mathcal{B}$ is the basic \emph{disjointness} relation
\end{itemize}
\par\vspace{5mm}
\noindent{\bf (Inheritance/Disjointness)} The \emph{inheritance} relation $\sqsubseteq^*\subseteq\mathcal{B}\times\mathcal{B}$ is the smallest such that:
\begin{itemize}
\item $P\sqsubseteq^*P$  (Reflexivity)
\item if $P\sqsubseteq Q$ and $Q\sqsubseteq^*R$ then $P\sqsubseteq^*R$ (Transitivity)
\end{itemize}
The \emph{disjointness} relation $\#^*\subseteq\mathcal{B}\times\mathcal{B}$ is the smallest such that:
\begin{itemize}
\item if $P\#Q$ or $Q\#P$ then $P\#^*Q$ (Symmetry)
\item if $P\sqsubseteq^*Q$ and $Q\#^*R$ then $P\#^*R$ (Chaining)
\end{itemize}
\par\vspace{5mm}
\noindent {\bf (Consistent Definition)}(cf. \cite{carpenter_inclusion_1991}, \emph{Conjunctive Concept}) A set $D\subseteq\mathcal{B}$ is a \emph{consistent definition} on $\mathcal{B}$ iff:
\begin{enumerate}
\item For all $x,y\in D$, it is not the case that $x\#y$.
\item For all $x\in D$, $y\in\mathcal{B}$, if $x\sqsubseteq y$, then $y\in D$.
\end{enumerate}
\par\vspace{3mm}
\noindent {\bf Lemma 3.1.} For $a,b\in\mathcal{B}$ such that it is not the case that $a\#b$, let $D_a=\{c\vert a\sqsubseteq c\}$ and $D_b=\{c\vert b\sqsubseteq c\}$. $D^*=D_a\cup D_b$ is a consistent definition on $\mathcal{B}$.

Proof. Because of the reflexivity of $\sqsubseteq$, it is obvious that  $D^*$ meets condition (2) above. Suppose (1) does not hold of $D^*$, i.e. there exists $x,y\in D^*$ such that $x\#y$. Then, by chaining we know that $a\#b$, since for all $x\in D^*$, either $a\sqsubseteq x$ or $b\sqsubseteq x$. But we have already said that it is not the case that $a\#b$, so (1) must hold of $D^*$. Therefore $D^*$ is a consistent definition on $\mathcal{B}$. $\quad\square$
\par\vspace{3mm}
\noindent {\bf Lemma 3.2}
Let $a,b\in\mathcal{B}$ be such that it is not the case that $a\sqsubseteq b$, and let $\mathcal{F}=\{D\vert D\text{ is a consistent definition}\}$. If $a$ is consistently definable, then there exists some $D_{\lnot b}^a\in\mathcal{F}$ such that $a\in D_{\lnot b}^a$ and $b\notin D_{\lnot b}^a$.

Proof. Assume $a$ is consistently definable. Then there exists some $D^a\in\mathcal{F}$ such that $a\in D^a$. Either $a\# b$ or not. Suppose $a\#b$. Then $b\notin D^a$, by condition (1) for consistent definition. Suppose it is not the case that $a\#b$. Then there is no relation between $a$ and $b$, which means that if $D^a$ is a consistent definition, then $D_{\lnot b}^a=D^a\setminus\{b\}$ is also a consistent definition. By the definition of $D_{\lnot b}^a$, $b\notin D_{\lnot b}^a$ and $a\in D_{\lnot b}^a$. $\quad\square$
\par\vspace{3mm}
\noindent{\bf The genus-differentia inheritance theorem (GDIT):} Given $\sqsubseteq $ and $\#$,  such that $\sqsubseteq $ is inclusion and $\#$ is disjointness over $\mathcal{B}$, a consistently definable set of atoms, there exists a set $\mathcal{F}$, a set $\Rep_\mathcal{B}\subseteq\mathcal{P}(\mathcal{F})$ (where $\mathcal{P}(\mathcal{F})$ is the power set of $\mathcal{F}$), and a function $i:\mathcal{B}\rightarrow\Rep_\mathcal{B}$ such that $a\sqsubseteq b\Leftrightarrow i(a)\subseteq i(b)$  and $a\#b\Leftrightarrow i(a)\cap i(b)=\emptyset$.

Proof. Let $\mathcal{F}=\{D\vert D\text{ is a consistent definition on }\mathcal{B}\}$, and let $i:\mathcal{B}\rightarrow\mathcal{P}(\mathcal{F})$ be a function such that $i(x)=\{D\in\mathcal{F}\vert x\in D\}$.

Assume $a\sqsubseteq b$. Suppose $D\in i(a)$. Then $a\in D$, by the definition of $i$, which means that $b\in D$ by (2). So $D\in i(b)$, by the definition of $i$. We therefore have $a\sqsubseteq b\Rightarrow i(a)\subseteq i(b)$.

Assume that it is not the case that $a\sqsubseteq b$. Then, by (3.2) there exists some $D^a_{\lnot b}\in\mathcal{F}$ such that $a\in D^a_{\lnot b}$ and $b\notin D^a_{\lnot b}$. Then $D^a_{\lnot b}\in i(a)$, but $D^a_{\lnot b}\notin i(b)$, which means that $i(a)\not\subseteq i(b)$. So we have: If it is not the case that $a\sqsubseteq b$, then it is not the case that $i(a)\subseteq i(b)$, which contraposits to $i(a)\subseteq i(b)\Rightarrow a\sqsubseteq b$.

We have therefore shown that $a\sqsubseteq b\Leftrightarrow i(a)\subseteq i(b)$, and we now turn to proving that $a\#b\Leftrightarrow i(a)\cap i(b)=\emptyset$.

Assume $a\#b$. Now suppose $i(a)\cap i(b)\neq\emptyset$. Then there exists some $D\in\mathcal{F}$ such that $a\in D$ and $b\in D$. But, by (1), this cannot be the case. So,  $a\#b\Rightarrow i(a)\cap i(b)=\emptyset$.

Assume $i(a)\cap i(b)=\emptyset$. Then there exists no $D\in\mathcal{F}$ such that $a\in D$ and $b\in D$. Now suppose it is not the case that $a\#b$. Let $D_a$, $D_b$, and $D^*$ be defined as in (3.1). Then  $D^*$ is a consistent definition on $\mathcal{B}$. But, by the reflexivity of $\sqsubseteq $, $a\in D_a$ and $b\in D_b$, which means that $a,b\in D^*$. So there is some $D\in\mathcal{F}$ such that $a\in D$ and $b\in D$, a contradiction. Therefore, $i(a)\cap i(b)=\emptyset\Rightarrow a\#b$.

We have therefore shown that $i(a)\cap i(b)=\emptyset\Leftrightarrow a\#b$. $\quad\square$

\clearpage
