\begin{center}
\Large \textbf{Chapter 1: Introduction} \\[1ex]
\end{center} 

\begin{abstract}
In theories of lexical decomposition, word meaning is analyzed as composed from some atomic units of semantic content. Most theorists seem to intend for their atoms to be interpreted as concepts, but none have provided a framework to fully develop this interpretation; it is assumed and used, but not justified. All theories of lexical decomposition organize semantic atoms in an inheritance hierarchy. In this thesis, I build on the mathematics of inheritance networks to develop a framework that is used to prove and analyze an isomorphism between semantic atoms and empirical observations of the hierarchical organization of concepts. The framework is presented not only as a means for justifying the conceptual interpretation, but also as a set of formal tools that might be used to facilitate future analysis in the domains of formal linguistics and cognitive science.
\end{abstract}

A theory of lexical decomposition is a semantic theory that analyzes word meaning as composed from some basic units of meaning, or \emph{atoms}, that are taken as primitive to the theory. Although decomposition is not without its opponents, notably Fodor and Lepore \cite{fodor_emptiness_1998} \cite{fodor_morphemes_2000} \cite{fodor_impossible_1999} \cite{fodor_impossible_2005}, it remains one of the most popular (if not \emph{the} most popular) approaches to lexical semantics. However, theories of lexical decomposition cannot do all of the work we require from a total theory of lexical semantics, or word meaning. David Lewis has observed that sentential decomposition cannot account for semantic content unless we augment the decompositional theory with some story about how atoms hook up with the world \cite{lewis_general_1970}. The same can be said for lexical decomposition. Despite this limitation of decompositional theories, there is an overwhelming consensus among cognitive scientists that decompositional models accurately reflect our cognitive representations of meaning, as Fodor and Lepore have themselves observed \cite{fodor_impossible_2005}.\footnote{Fodor and Lepore express puzzlement at this consensus, on the grounds that ``as far as [they] can tell, there is practically no evidence to support it'' \cite[pg.\ 1]{fodor_impossible_2005}. However, they imply that they will only accept as evidence an argument that it is impossible to represent meaning in any manner other than decompositionally: ``For example, there is no scientific evidence that you \emph{can't} have a word that expresses the concept BREAK$_\text{TR}$ unless you have the concept CAUSE. But there ought to be if CAUSE is a constituent of BREAK$_\text{TR}$'' \cite[pg.\ 1, emphasis added]{fodor_impossible_2005}. But this places too high a standard of justification for the empirical claims that cognitive scientists make; they require only that we \emph{do} represent meaning decompositionally, not that we do so \emph{necessarily}.} Given this consensus, we should not hastily abandon lexical decomposition, but should instead look for some interpretation of the atoms that will fortify theories of lexical decomposition by providing them with a source of semantic content. The aim of this thesis to interpret semantic atoms as concepts.

The present chapter is introductory. I will set some expectations for what will be found later in the thesis, explain some preliminary ideas that will be used throughout the thesis, and define some basic notation. Much of this content will be repeated later.

Chapter 2 will provide suitable background material to understand the relevant aspects of theories of lexical decomposition. In Section \ref{sec2.1}, Chapter 2, I will survey theories of lexical decomposition authored by Ray Jackendoff \cite{jackendoff_semantics_1983} and Beth Levin \& Malka Rappaport Hovav \cite{levin_building_1998}, in order to identify semantic atoms as a formal representation of \emph{semantic content} and \emph{inheritance networks} as a set of formal relations that is used to model relations between semantic atoms. By \emph{semantic content}, I mean the \emph{informational}\marginpar{The use of italics here is not parallel. Does it have to be? I'm not using the italics to draw a parallel, but two flag two terms to take note of, because they are the terms I'm about to spend the next two paragraphs defining.} part of semantics, which includes not only reference, but other information which native speakers of a language will tend to associate with reference. For example, the atom {\bf human} refers to humans, so these are included in the content provided by {\bf human}. Native speakers of English will tend to recognize that humans reproduce sexually, walk on two legs, build skyscrapers, cook dinner, and use language. While ``human'' does not refer to reproducing sexually, etc., this is all part of the semantic information provided by the atom {\bf human}. \emph{Semantic content} is distinguished from semantic structure, which is the way in which we situate atoms within the total semantics of a word. For example, we might wish to distinguish between essential and non-essential properties of humans.\footnote{Theories that organize semantic content in this way are called \emph{essentialist} theories. Essentialist theories are outside the scope of this paper and are only mentioned here as a way of grasping the difference between semantic content and semantic structure. The structural aspect of essentialism is much easier to intuit by means of a short explanation than the structures that will be seen later.} Labelling some properties as essential is an aspect of semantic structure, not semantic content.

An \emph{inheritance network} is an abstract system in which mathematical \emph{types} are organized by the reflexive, transitive inheritance relation. A \emph{type} is a purely abstract mathematical object, studied by \emph{type theory}, that is in some ways analogous to the sets studied by set theory. Where sets have members, types have tokens, but tokens are not \emph{in} the type in the way that members are in sets. A token of a type has constraints placed on it by the type without being \emph{in} the type. In fact, since types are entirely basic, it doesn't make sense to talk about things being \emph{in} a type at all. We will largely be unconcerned with the mathematics of type theory in this thesis. In all of the contexts that types will be used here, they are taken as models of semantic atoms, so for practical purposes, the terms will be somewhat interchangeable in that they have the same extension, but different intensions. ``Type'' will always refer to semantic atoms, but I will refer to atoms as ``types'' when taking a mathematical perspective on the atoms. The distinction is subtle and can be ignored by most readers.

Chapter 2 will close by giving a slightly more detailed presentation (in comparison to the presentation of other theories of lexical decomposition) of James Pustejovsky's \emph{Generative Lexicon} \cite{pustejovsky_generative_1998}. Pustejovsky's system will later serve as a case study in how the conceptual interpretation of semantic atoms can be applied to practical questions.

Having established that inheritance networks appear in all theories of lexical decomposition, in Chapter 3, I will turn toward empirical research on concepts. This chapter will be a somewhat historical story about the downfall of the \emph{classical view of concepts}. I will identify two claims made by the classical view, one pertaining to the intrinsic content of concepts, i.e.\ that definitions provide necessary and sufficient conditions for category membership, and the other pertaining to the relations that hold between concepts. i.e.\ that concepts are organized hierarchically. It will be argued that philosophers and cognitive scientists, at some point, lost sight of the relational claim, so that when empirical research conclusively established that the intrinsic claim is false, the classical view came to be almost universally rejected. However, as will be seen, the same research that establishes that category membership is not determined by necessary and sufficient conditions also establishes that concepts are organized hierarcically, so there remains an aspect of the classical view that has not been overturned. Both of these empirical findings will serve to advance the conceptual interpretation of semantic atoms. Cognitive scientists have found that category membership is determined by overall similarity, or \emph{typicality}, and it will turn out that inheritance is a typicality relation. The inheritance networks of decompositional theories are simply a way of hierarchically organizing semantic atoms according to the inheritance relation.

Chapter 4 comprises the main theoretical material of the thesis. I will give formal definitions of \emph{inheritance}, \emph{inheritance network}, \emph{consistent definition}, and \emph{functional role}. Inheritance and inheritance networks have already been mentioned, but a few words should be said here about the other terms, which are original to this thesis. \emph{Consistent definition} is an abstract set-theoretic structure defined over the atoms included in an inheritance network, which serves as a mathematical model of the genus-differentia definitions in the classical view of concepts. The \emph{functional role} of an atom is the set of all consistent definitions containing the atom. In the climax, a theorem will then be presented\footnote{The proof of the theorem is not included in Chapter 4, but in an appendix.} stating that atoms in an inheritance network are isomorphic to functional roles. An \emph{isomorphism} is a relation-preserving bijection between two domains. The relations preserved in this isomorphism are the type-theoretic inheritance relation and the subset relation between functional roles. Less formally, some philosophical consequences of the theorem will be explained.

In Chapter 5, I will apply the mathematical framework of Chapter 4 to James Pustejovsky's \emph{Generative Lexicon}, in order to illustrate why an isormophism is useful for understanding concepts. Some theoretical questions about identity between concepts will be raised, and a diagnostic for non-identity will be introduced and applied to answer these questions. Questions about the practical consequences of the theorem for lexical decomposition will be raised, but an answer will only be given in the broadest strokes; a detailed answer to these questions is outside the scope of this thesis.

In Chapter 6, I will summarize the results of the thesis and point toward some outstanding questions and directions for future research.

Before continuing, it will be useful to define some notation that will be used throughout the thesis. {\bf Boldface type} will be used for types, semantic atoms, and concepts; since the point of this thesis is that, in the context of a theory of lexical decomposition, types \emph{are} semantic atoms and semantic atoms \emph{are} concepts,\marginpar{I've already qualified this with "in the context of a theory of lexical decomposition." I'm not married to that language, if you have some other suggestion that you prefer.} I annotate them all in the same way. Words will be mentioned using the standard convention of quotes, e.g.\ ``begin'' is a verb. The inheritance relation will be represented by the symbol $\sqsubseteq$. Another relation---\emph{disjointness}---is included in the inheritance networks studied here, but is not presented as a direct model of conceptual hierarchy. Disjointness is necessary for establishing the theorem of Chapter 4, and it does reflect other properties of semantic content. Disjointness will be represented by the symbol $\#$. The notation $\alpha :\beta$ will indicate that $\alpha$ is of type $\beta$; when $\alpha$ is a type, this notation is equivalent to $\alpha\sqsubseteq\beta$, with the difference being that $\alpha$ need not be a type in order for the sentence $\alpha:\beta$ to be well-formed. The usual entailment symbol $\vdash$ is used in this thesis to mean specifically entailment within the logic of type-inheritance, which is a very different system from the more-familiar first-order logic. The details of this logic are not important for the purposes of this thesis, but I will at times refer to entailment relations that hold within the logic of type-inheritance. Graphically, inheritance will be represented by an arrow, with $\alpha\rightarrow\beta$ equivalent to $\alpha\sqsubseteq\beta$; disjointness is graphically represented by a dashed line. Graphs of inheritance networks will sometimes be color-coded, and a code will be defined in the text where one is used.

\clearpage
