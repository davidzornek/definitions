\begin{center}
\Large \textbf{Chapter 6: Conclusion} \\[1ex]
\end{center} 

The argument presented here can be summarized as follows:\footnote{This is a \emph{logical} reconstruction of the argument offered in this thesis, and it does not map exactly onto the order in which premises were presented.}
\begin{enumerate}[(i)]
\item Theories of lexical decomposition break word meaning down into primitive semantic atoms organized in a type-inheritance network.
\item If advocates of lexical decomposition wish to have a complete, coherent account of lexical semantics, the atoms must be interpreted in some way that explains how they hook up with the world to acquire their content.
\item Any viable interpretation will explain why type-inheritance networks appear in all theories of lexical decomposition.
\item Interpreting atoms as concepts explains why type-inheritance networks appear in all theories of lexical decomposition.
	\begin{enumerate}[(a)]
	\item The classical view of concepts holds that genus-differentia definitions (I) provide necessary and sufficient conditions for category membership, while also (R) positioning concepts within a hierarchy.
	\item Empirical research has established that category membership is determined by typicality, rather than necessary and sufficient conditions.
	\item Empirical research has also established that concepts are organized hierarchically.
	\item Inheritance networks are a model of conceptual hierarchy that accounts for category membership determined by typicality.
	\end{enumerate}
\item It is generally agreed that concepts hook up with the world in some way.
\item Therefore, concepts are a viable interpretation of semantic atoms that will allow advocates of lexical decomposition to move toward a complete, coherent account of lexical semantics.
\end{enumerate}

The framework presented here only models two semantic relations, namely disjointness and the relation $\Delta$. But many other semantic ordering relations exist, e.g.\ meronymy and holonymy (relations between parts and a whole). If we wish to use inheritance networks as a total model of lexical semantics, we will have to add models of these other relations to the network.

Nevertheless, there are some important applications of the theory as it stands. In particular, the theory has been used to answer some questions about whether two instances of a word correspond to the same or distinct concepts. This was done by means of a diagnostic that trades on the observation that functional roles provide a necessary condition for identity between concepts. This framework, then, provides a set of mathematical tools that can be applied to analysis of concepts, and future research can perhaps identify additional diagnostics or other applications of the framework to different kinds of conceptual questions.

Many theories of lexical decomposition leave wide open the question of which atoms should be included in an inheritance network and which inheritance relations hold between them. Very little guidance is given within the theories about how to go about identifying an appropriate inheritance network to use. This vagueness carries over into technology: computational resources such as VerbNet are based explicitly on theories of lexical decomposition. There is significant debate among researchers who work on these implementations regarding when to \emph{lump} two classes together into the same type and when to \emph{split} them into distinct types. By interpreting types as concepts, we have a well-justified criterion for determining, at least, when to split. When a diagnostic shows that types correspond to distinct concepts, we should regard them as separate atoms in the inheritance network, and therefore it stands to reason that we should split them in the computational lexicon (issues pertaining to the computational efficiency of the implementation notwithstanding).

This thesis was facilitated by proving that inhertance networks are isomorphic to a new mathematical structure I have called the \emph{functional role} of an atom. Functional roles provide a necessary condition for identity between concepts, but they do not provide a sufficient condition. In order to complete the conceptual interpretation of atoms, more research must be done to discover a sufficient condition. I am not prepared to offer anything resembling a sufficient condition, and there is no reason to suppose that one will so neatly fall out of defining a new mathematical concept as we have seen here. Concepts appear to be inherently fuzzy at the edges. Until we are able to better understand this fuzziness, it would seem, a clear sufficient condition for identity of concepts will have to lie in wait. It may be that no such condition is possible. After all, this question is closely allied to the question, ``What is a member of the category of concepts?'' But {\bf concept} is itself a concept, and it is an important part of the theory presented here that the kinds of necessary and sufficient conditions we are asking for do not, in general, exist. 

\clearpage
