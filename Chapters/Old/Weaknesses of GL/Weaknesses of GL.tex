
\documentclass[12pt]{amsart}
\usepackage{geometry} % see geometry.pdf on how to lay out the page. There's lots.
\geometry{a4paper} % or letter or a5paper or ... etc
\input xy
\xyoption{all}
\usepackage{bussproofs}

% See the ``Article customise'' template for come common customisations

\title{What the Generative Lexicon Can't Do \\ That Concepts Can}
\author{David Zornek}
\date{} % delete this line to display the current date

%%% BEGIN DOCUMENT
\begin{document}

\maketitle

The GL affords us a powerful and efficient formal modal of polysemy, but it can not do all of the work we might want from a full theory of lexical semantics. Since the GL relies entirely on being able to make use of an inheritance lattice $\mathcal{I}$, questions pertaining to $\mathcal{I}$ other than its application within GL are necessarily outside of the theory. In particular, we might want to ask: Where does $\mathcal{I}$ come from? Is there only one right $\mathcal{I}$, or are there number of different acceptable $\mathcal{I}$s? If more than one acceptable $\mathcal{I}$ exists, how might we choose between $\mathcal{I}$s in representing the semantics of a given lexical item? We can list as many questions about $\mathcal{I}$ as we like, and certainly some have already been answered on purely formal terms, in particular by Bob Carpenter \cite{Carpenter92}. It seems to me that the question of where $\mathcal{I}$ comes from is second only to the question of what $\mathcal{I}$ is in terms of fundamentality, however, and this, I believe, has not been and cannot be answered on purely formal terms.

Another problem that GL cannot solve is the fixation of reference. GL is a purely abstract system of types, predicates and logical/functional relations that does not involve the real-world object, events, what-have-you that we seem to be talking \emph{about} when we use language. Even Frege, among whose most famous achievements is the recognition that linguistic meaning is not entirely determined by reference, did not deny that reference is a crucial element in lexical meaning.\footnote{In a letter to Husserl \cite{Frege1891}, Frege himself regards real-world objects as the end-of-the-line in lexical meaning, and in fact the point of the letter is to communicate to Husserl that a difference in how the two men ``get at'' real-world objects is of fundamental importance. Also, one very common characterization of \emph{sense} among Frege scholars is that it is to be regarded as a ``route to the reference.''} It seems reasonable, then, to search for a way of grounding the GL in the real-world reference of words.

It is my hope that, by introducing a formal account of the process of forming a new \emph{concept} into our semantic toolbox, we will be able to situate the GL as part of a wider theory of lexical semantics, in particular where the above difficulties are concerned.

\section{The $\mathcal{I}$ Problem}

The $\mathcal{I}$ problem is primarily of internal importance to the GL. I say that it is primarily of internal importance because of the fact that, not only is $\mathcal{I}$ an element of the quadruple that GL takes as a specification of the semantics of a lexical item, but the types used in $\mathcal{A}$, $\mathcal{E}$, and $\mathcal{Q}$ are provided by $\mathcal{I}$. In order for us to have $\text{TELIC}=\textbf{read}(x,y)$ in our qualia structure for ``book,'' the type {\bf read} must be available in $\mathcal{I}$. Moreover, at least some of the generative rules, e.g. subtype coercion, rely on an inheritance relation existing in $\mathcal{I}$; we are only able to coerce the type of ``Honda'' from {\bf car} to {\bf vehicle} when {\bf car} is given as a subtype of {\bf vehicle} in $\mathcal{I}$. However, if we choose some system of lexical semantics other than GL, questions pertaining to $\mathcal{I}$ are less important; in any system that does not rely on a type lattice, questions pertaining to the type lattice simply will not arise.

Many psychologists who study concepts have observed that, particularly for object concepts, subjects tend toward a hierarchical organization of concepts, much like the type lattice required by GL. Eleanor Rosch \cite{Rosch78} has argued in favor of a \emph{basic level of categorization}, which serves as a conceptual starting point; all other categories, both subordinate and superordinate, are abstractions away from this basic level. Her criteria of basicality have been found to be incomplete with respect to explanation of and predictability for later data; Murphy and Brownell \cite{MurphyBrownell85} have offered a revised set of criteria for basicality which do better. The details of their basicality criteria will merit independent treatment. What is important at present is that literature in the psychology of concepts give us an answer for the question of where $\mathcal{I}$ comes from: it is a reflection of the hierarchical organization of concepts.

Most of the literature on concepts focuses on object concepts, and therefore it will be most directly useful in giving insight into the portion of $\mathcal{I}$ that is concerned with subtypes of {\bf object} or {\bf things} (or whatever we decide to call this basic type). The main focus of this project will be nominals, object concepts and object types, but some attempt will have to be given at explaining whether and how what is said in this domain extends to other domains. This will be particularly important where event and property types are concerned, since these types are used in the typed feature structures of nominals.

One difficulty that will have to be faced in using the hierarchical organization of concepts to explain $\mathcal{I}$ is that $\mathcal{I}$ requires clear and definite hierarchy relations, while there is other research on concepts showing that subjects do not always organize their concepts hierarchically (or even that they rarely do in practice). This is a significant problem, to which I have no definite answer at present. However, I do have one idea for how a solution might be obtained. We have observed that the GL takes a ``god's eye'' view on lexical semantics, rather than giving an account of how individuals actually represent word meanings. By keeping this in mind, we might be able to free ourselves from committing to the view that language-users actually possess a static all-or-nothing hierarchy in their representations of concepts. Such a hierarchy of concepts might give us a ``god's eye'' view on concepts in exactly the way that GL would require. Moreover, it may turn out that, in looking at research on conceptual hierarchies, we are able to discover when subject do and do not organize their representations hierarchically and use this as a way of relating the GL to other typed feature semantics which, instead of taking a ``god's eye'' view on meaning, are concerned with individual representations of meaning, such as in Jackendoff's \emph{Semantics and Cognition} \cite{Jackendoff83}.

\section{The reference problem}

The reference problem is primarily of external importance to the GL. It is not the goal of GL to fix a real-world reference for words. Therefore, so far as GL is concerned, how and whether we attach a reference to words is a problem for other theories to address. However, if we take a wider view of the project as contributing to a fuller theory of lexical semantics, it seems as though we should want some way of relating a word to its reference.

There are two major classes of experiments on concepts, which we might call \emph{category-learning experiments} and \emph{concept-formation experiments}. On a more abstract theoretical level, the former are experiments in which subjects are given some task during which they are to correctly categorize things into categories pre-determined by the researcher. In the latter, there is no pre-determined category; the subject are to create their own. More concretely, it seems as though we can say that, in category-learning experiments, subjects are given feedback based on whether the content of their category matches up with the researcher's predetermined category. In concept-formation experiments, no such feedback is given. Although the more concrete way of distinguishing between the two types of experiments will be useful in determining which class a given piece of research falls into, it is the more abstract characterization that more clearly relates to the reference problem. In the process of category-learning, whether an item does or does not belong in the category has already been decided; the goal is for subjects to correctly identify membership. In the process of concept-formation, the task is precisely to decide on membership; in concept-formation, subjects perform a task which creates the membership relation between some items and a category. If the category is a concept associated with some word, it is in the process of concept-formation that the reference of a word gets fixed.

If we are able to give some formal description of the process of concept-formation, we can then ask how and whether there is some relation between this formal description and the typed feature structures of GL. If we are able to relate the two (and in fact, the best strategy will be to formalize concept-formation in a way that admits of relation to GL), we will have some account of how words are related to reference. Specifically, the word-reference relation will be mediated by the concept.

For this reason, I propose to focuse mainly on concept-formation experiments, although something will be said about how the results of category-learning experiments relate to what is done here.

Psychologists disagree on whether the hierarchical organization of concepts is a byproduct of concept-formation or some process of inference that occurs after concepts are formed. I need to look at this literature more closely.\footnote{In fact, I have not yet looked into this literature at all. I am only aware that the debate exists.} My hope is that I can find some justification for siding with those who hold that hierarchies are a byproduct of concept-formation, since this will provide an elegant sort of unity to the solutions of both problems.


\end{document}